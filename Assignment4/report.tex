\documentclass[11pt]{article}
\usepackage[utf8]{inputenc}
\usepackage{geometry}
\geometry{a4paper}

\usepackage{graphicx}
%\usepackage[parfill]{parskip}
\usepackage{bbm}
\usepackage{amssymb}
\usepackage{amsmath}
\usepackage{amsfonts}
\usepackage{color}
\usepackage{hyperref}

\title{Assignment 4 Internet Programming}
\author{Marc Went (2507013) and Ferry Avis (1945653)}

\begin{document}
\maketitle
\section{A paper storage server}

The assignment description is very ambiguous. Several statements can be interpreted in multiple ways. One example is that the template contains a `hotel', `paper' and `web' folder. Initially the thought was that all the paper code - cgi as well as the php must be in the projects folder, so all the paper storage code in the paper folder. Further more there is a `cgi-bin' folder within the `web' folder.
This is fixed by having the `Makefile' create symbolic links to the web folder. As well as allowing Apache to follow these symlinks. 

\section{A hotel reservation server}

PHP's \texttt{socket\_read} is normally a blocking call until the specified byte length has been received. The problem is that the message length is not known in advance. By using the \texttt{PHP\_NORMALLY\_READ} parameter, the call blocks until a carriage return or new line has been found. This is okay for output of commands, but there is no new line or carriage return after \texttt{hotelgw}. This causes the PHP script to hang after the output of the request has been received. To circumvent this, before invoking the \texttt{socket\_read} a \texttt{socket\_recv} with the peek parameter is obtained to look if the data coming next is output of the issued command or if it is the prompt. If it is the prompt, a standard \texttt{socket\_read} is called with the length of the prompt to remove it from the buffer. If the socket is closed before emptying the read buffer, we get an error at the gateway, as the socket as been closed prematurely.

\section{Answers to questions}

\subsection{Question A}

An alternative would be to execute the code from a PHP script with an \texttt{exec} command. Where the compiled C-code could be executed. The C-code is still necessary as middle ware between Web code (PHP, HTML, JS) and the paper server (C Sun-RPC). Another thing that would need to change is the assignment of the port number. Currently the Sun-RPC code does not define a connecting port since it "knows". 

\subsection{Question B}

Pros: It is very fast to display the data on a webpage since JSON needs to be parsed as well as displayed.
Cons: When using the program for other means than a webpage, parsing JSON is easier and more standardized than a HTML page that can be displayed in many different ways.

\subsection{Question C}

We don't think it is possible since binary files over GET would show in the address bar and possibly crash browsers. If it were possible POST would still be more favorable than GET since POST data can be encoded in a way binary data is not lost. GET is considered `safe' for binary file transfer but even so the address bar would be filled with the file size worth of data. Which can be server megabytes big.

\subsection{Question D}

The advantage of using the POST/Redirect/GET pattern is that if users refresh the page after they have send the form, the request isn't send again. This can possible lead to double bookings in case of the hotel reservation system. By redirecting the user if the POST request has been made to another page printing the success message, this is avoided. In most cases, GET requests are idempotent, while POST requests are not.

\end{document}
